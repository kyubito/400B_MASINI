\documentclass{article}
\usepackage[utf8]{inputenc}

\title{Research Assignment 2}
\author{Adrien Masini }
\date{March 2020}

\documentclass{article}
\usepackage{graphicx}
\graphicspath{ {./images/} }

\begin{document}
\maketitle
\section{Introduction}

1.
We are going to study the behavior of the dark matter halo of M33 due to tidal forces caused by MW and M31 and later by the MW-M31 merger. In details, we will observe and map the evolution of dark matter through the process and how dark matter evolve regarding internal mass loss and its rate.
\vspace{5mm}

2.
Dark matter is a physical behavior of the universe that we have yet to understand. This topic will help us understand how dark matter behave and will give us more information about its properties. Also, it will give more details about galaxy formation, there different shape and there evolution over time.
\vspace{5mm}

3.
We know almost nothing about the behavior of Dark Matter except that even though we cannot see it, it interacts with normal matter via gravity.
\vspace{5mm}

4.	
What is Dark Matter ? How did galaxies formed and what is the role of dark matter in their formation ? How dark matter influence the shape of a galaxy ?
\vspace{5mm}

5.

@article{Wechsler_2018,
   title={The Connection Between Galaxies and Their Dark Matter Halos},
   volume={56},
   ISSN={1545-4282},
   url={http://dx.doi.org/10.1146/annurev-astro-081817-051756},
   DOI={10.1146/annurev-astro-081817-051756},
   number={1},
   journal={Annual Review of Astronomy and Astrophysics},
   publisher={Annual Reviews},
   author={Wechsler, Risa H. and Tinker, Jeremy L.},
   year={2018},
   month={Sep},
   pages={435–487}
}

\vspace{5mm}

@article{Green_2019,
   title={The tidal evolution of dark matter substructure – I. subhalo density profiles},
   volume={490},
   ISSN={1365-2966},
   url={http://dx.doi.org/10.1093/mnras/stz2767},
   DOI={10.1093/mnras/stz2767},
   number={2},
   journal={Monthly Notices of the Royal Astronomical Society},
   publisher={Oxford University Press (OUP)},
   author={Green, Sheridan B and van den Bosch, Frank C},
   year={2019},
   month={Oct},
   pages={2091–2101}
}

\vspace{5mm}

@article{Delos_2019,
   title={Tidal evolution of dark matter annihilation rates in subhalos},
   volume={100},
   ISSN={2470-0029},
   url={http://dx.doi.org/10.1103/PhysRevD.100.063505},
   DOI={10.1103/physrevd.100.063505},
   number={6},
   journal={Physical Review D},
   publisher={American Physical Society (APS)},
   author={Delos, M. Sten},
   year={2019},
   month={Sep}
}

\vspace{5mm}

6.
\begin{figure}[htp]
    \centering
    \includegraphics[width=7cm]{Density Profile Evolution Of Halo.jpeg}
    \caption{Density Profile Evolution Of Halo}
    \label{fig:galaxy}
\end{figure}

\vspace{5mm}

\maketitle
\section{Proposal :}

\vspace{5mm}

1.
Scientist in the field think that the baryonnic matter has a direct impact in the density profile of dark matter halo. From "On the average density profile of dark-matter halos in the inner regions of massive early-type galaxies" by Claudio Grillo published in arvix.org on February 16th 2012.
\vspace{5mm}

2.
I will use the data we have on halo particle for M33 and see how they evolve over some timestep. I will provide a facticious 3D space where the particle around will move away or toward it depending on they velocity. Then we would be able to observe an evolution in the density of that space.
\vspace{5mm}

3.
\vspace{5mm}

4.
I think the density of the halo should decrease until M33 collides with the new MW-M31 merger system and then the two halos will add their own Dark Matter to form a new more massive halo of Dark Matter.












\end{document}
